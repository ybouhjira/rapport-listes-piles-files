\documentclass[a4paper]{article}

\usepackage[francais]{babel}
\usepackage{xltxtra,fontspec,xunicode}

\usepackage{listings}

\usepackage{graphicx}
\usepackage{float}


\usepackage{listings}
\lstset{                                                                          
   language=C,
   basicstyle=\ttfamily,                                                           
   frame=lrtb ,
   columns=fullflexible      
}

%space paragraphs
\setlength{\parskip}{\baselineskip}%

\begin{document}

\tableofcontents

\section{Introduction}
Le but de ce TP est de réaliser une implémentation des structures de données qui
sont les listes, les piles et les files en language C à l'aide des pointeurs.

Pour chaque structure de données on devra être capable d'insérer, ou de supprimer
un élément, ou encore d'afficher, copier ou savoir la taille de la structure.

\section{Les listes simplement chainées}

\subsection{Structure de données}

\begin{lstlisting}
typedef struct Liste 
{
  struct Liste *suiv;
  int val;
} Liste;
\end{lstlisting}

Cette structure reprèsente un élément de la liste chainée et contient deux variables: 

\begin{itemize}
  \item \textbf{suiv}: L'élément suivant cet élément.

  \item \textbf{val}: La valeur contenu dans cet élément.
\end{itemize}

\subsection{Fonctions}
\subsubsection{Insértion en tête}

\begin{description}
  \item[Nom de la fonction:] liste\_ajout\_debut

  \item[Entrées:] \hfill
    \begin{itemize}
      \item Liste **liste : La liste dont laquelle on va insérer
      \item int val : La valeur à insérer
    \end{itemize}

  \item[Description:] \hfill \\ 
    Insére l'élément au début de la liste, en remplacant la tête de la liste 
    par le nouvel élément.

  \item[Code:] \hfill
  \begin{lstlisting}
void liste_ajout_debut(Liste **liste, int val)
{
  Liste *nouveau = liste_creer(val);
  if(*liste)
    {
      Liste *tete = *liste;
      nouveau->suiv = tete;
    }
  *liste = nouveau;
}
  \end{lstlisting}
\end{description}

\subsubsection{Insértion en queue}

\begin{description}
  \item[Nom de la fonction:] liste\_ajout\_fin

  \item[Entrées:] \hfill
    \begin{itemize}
      \item Liste **liste : La liste dont laquelle on va insérer
      \item int val : La valeur à insérer
    \end{itemize}

  \item[Description:] \hfill \\ 
    Insére l'élément au début de la liste, en parcourant la liste jusqu'à la
    fin et en attachant le nouveal élément à la queue de la liste.


  \item[Code:] \hfill 
  \begin{lstlisting}
void liste_ajout_fin(Liste **liste, int val)
{
  Liste *courant = *liste;
  Liste *nouveau = liste_creer(val);
  if(!(*liste)) *liste = nouveau;
  else
    {
      for(; courant->suiv; courant = courant->suiv);
      courant->suiv = nouveau;
    }
}
  \end{lstlisting}
\end{description}


\subsubsection{Insértion dans la liste}
\begin{description}
  \item[Nom de la fonction:] liste\_inserer

  \item[Entrées:] \hfill
    \begin{itemize}
      \item Liste **liste : La liste dont laquelle on va insérer
      \item int val : La valeur à insérer
      \item int pos : La position de l'insertion
    \end{itemize}

  \item[Description:] \hfill \\ 
    Insert un élément dans la position donnée, en parcourant la liste et incrémentant
    un compteur qui permet de savoir si on est arrivé à la bonne position.
    
  \item[Code:] \hfill 
  \begin{lstlisting}
void liste_inserer(Liste **liste, int val, int pos)
{
  // Conditions
  assert(pos >= 0);

  // Nouvel élément pour la valeur
  Liste *nouveau = liste_creer(val);

  // Insértion
  if(pos == 0)
    {
      liste_ajout_debut(liste, val);
    }
  else if(!(*liste))
    {
      assert(0); // Erreur : liste vide pos doit etre 0
    }
  else
    {
      Liste *courant = *liste;
      int i;
      for(i = 0; i < pos - 1 && courant->suiv; ++i)
        courant = courant->suiv;
      if(i != pos - 1) assert(0); // pos > taille
      nouveau->suiv = courant->suiv;
      courant->suiv = nouveau;
    }
}
  \end{lstlisting}
\end{description}


\subsubsection{Suppression d'un élement de la liste}
\begin{description}
  \item[Nom de la fonction:] liste\_supprimer

  \item[Entrées:] \hfill
    \begin{itemize}
      \item Liste **liste : La liste de laquelle on va supprimer.
      \item int pos : La position de la suppression.
    \end{itemize}

  \item[Description:] \hfill \\ 
    Supprime l'élément qui se trouve à la position indiquée, en parcourant la 
    liste jusqu'à cette position, et en détachant l'élément de la liste.


  \item[Code:] \hfill \\
  \begin{lstlisting}
void liste_supprimer(Liste **liste, int pos)
{
  assert(pos >= 0);
  assert(*liste);

  Liste *elem;
  if(!pos)
    {
      elem = *liste;
      *liste = (*liste)->suiv;
    }
  else
    {
      int i;
      Liste *courant = *liste;
      for (i = 0; i < pos - 1 && courant->suiv; ++i)
        courant = courant->suiv;
      if(i != pos - 1) assert(0); // pos > taille - 1
      elem = courant->suiv;
      courant->suiv = courant->suiv->suiv;
    }

  // Suppression de la mémoire allouée
  free(elem);
}
  \end{lstlisting}
\end{description}


\subsubsection{Copie de la liste}
\begin{description}
  \item[Nom de la fonction:] liste\_copier

  \item[Entrées:] \hfill
    \begin{itemize}
      \item Liste **liste : Une liste
    \end{itemize}

  \item[Sortie:] \hfill
    Une copie de la liste.

  \item[Description:] \hfill \\ 
    Parcours la liste en ajoutant ses éléments dans une seconde liste, et 
    retourne cette dernière.
    
  \item[Code:] \hfill
  \begin{lstlisting}
Liste* liste_copier(Liste *liste)
{
  Liste *ptr = liste, *copie = NULL, *ptrCopie = NULL;

  for(; ptr; ptr = ptr->suiv) {
      if(!copie)
        {
          ptrCopie = copie = liste_creer(ptr->val);
        }
      else
        {
          ptrCopie->suiv = liste_creer(ptr->val);
          ptrCopie = ptrCopie->suiv;
        }
    }

  return copie;
}
  \end{lstlisting}
\end{description}

\subsubsection{Affichage de la liste}
\begin{description}
  \item[Nom de la fonction:] liste\_afficher

  \item[Entrées:] \hfill
    \begin{itemize}
      \item Liste *liste : Une liste
    \end{itemize}

  \item[Description:] \hfill \\ 
    Affiche la liste.

  \item[Code:] \hfill 
  \begin{lstlisting}
void liste_afficher(Liste *liste, Liste *fin)
{
  Liste *courant = liste;
  printf("[");
  while (courant && courant != fin)
    {
      printf("%d",courant->val);
      courant = courant->suiv;
      if(courant && courant != fin) printf(", ");
    }
  printf("]");
}
  \end{lstlisting}

\end{description}


\section{Les piles}
\subsection{Structure de données}
La même que les listes chainées.

\begin{lstlisting}
typedef Liste Pile;
\end{lstlisting}

\subsection{Fonctions}
\subsubsection{Empilement}
\textbf{Nom de la fonttion:} pile\_empiler \\

Utilise la fonction d'ajout en tête d'une liste chainée.
\begin{lstlisting}
#define pile_empiler liste_ajout_debut
\end{lstlisting}

\subsubsection{Dépilement}
\begin{description}
  \item[Nom de la fonction:] pile\_depiler

  \item[Entrées:] \hfill
    \begin{itemize}
      \item Pile **pile : Une pile non vide
    \end{itemize}

  \item[Sortie:] \hfill
    La valeur supprimée.

  \item[Description:] \hfill \\ 
    Supprime l'élément en tête de pile (en utiliant la fonction de suppression en tête de liste)
    et renvoie ça valeur.

  \item[Code:] \hfill
  \begin{lstlisting}
int pile_depiler(Pile **pile)
{
  assert(*pile);

  int valeur = (*pile)->val;
  liste_supprimer(pile, 0);
  return valeur;
}
  \end{lstlisting}
\end{description}


\subsubsection{Copie de la pile}
\begin{description}
  \item[Nom de la fonction:] pile\_copier

  \item[Entrées:] \hfill
    \begin{itemize}
      \item Pile **pile : Une pile
    \end{itemize}

  \item[Sortie:] \hfill
    Pile* : Copie de la pile.

  \item[Description:] \hfill \\ 
    Dépile tous les élements de la pile et les empile dans deux autres piles.
    L'une sera retournée et on change le pointeur ``pile'' pour pointer 
    sur l'autre.

  \item[Code:] \hfill
  \begin{lstlisting}
Pile* pile_copier(Pile **pile)
{
  Pile *copie = NULL, *pile2 = NULL;

  while(*pile)
    {
      int val = pile_depiler(pile);
      pile_empiler(&pile2, val);
      pile_empiler(&copie, val);
    }

  *pile = pile2;

  return copie;
}
  \end{lstlisting}

\end{description}


\subsubsection{Affichage de la pile}
\begin{description}
  \item[Nom de la fonction:] pile\_afficher

  \item[Entrées:] \hfill
    \begin{itemize}
      \item Pile **pile : Une pile
    \end{itemize}

  \item[Description:] \hfill \\ 
    Dépile et affiche les éléments de la pile un par un dans une autre pile.
    Puis change le pointeur ``pile'' pour qu'il pointe sur cette dernière.

  \item[Code:] \hfill 
  \begin{lstlisting}
void pile_afficher(Pile **pile)
{
  Pile *copie = NULL;
  while(*pile)
    pile_empiler(&copie, pile_depiler(pile));

  int i = 0;
  while(copie)
    {
      int val = pile_depiler(&copie);
      printf("%d : %d\n", i++, val);
      pile_empiler(pile, val);
    }
}
  \end{lstlisting}

\end{description}
\section{Les files}
\subsection{Structure de données}
\begin{lstlisting}
typedef Liste FileElem;

typedef struct File
{
  FileElem *tete, *queue;
}File;
\end{lstlisting}

Un élement de la file à la même structure que selui de la liste, avec ses deux
variables \textbf{suiv} et \textbf{val}.

On déclare un structure \textbf{File} qui reprèsente la file et contient ça
tête et ça queue.

\subsection{Fonctions}
\subsubsection{Enfilement}

\begin{description}
  \item[Nom de la fonction:] file\_enfiler

  \item[Entrées:] \hfill
    \begin{itemize}
      \item File *file: La file
      \item int val : La valeur à enfiler
    \end{itemize}

  \item[Description:] \hfill \\ 
    Ajoute un élèment à la queue de la file. Si la pile est vide La tête et la
    queue vont pointer sur la même élément.

  \item[Code:] \hfill 
  \begin{lstlisting}
void file_enfiler(File *file, int val)
{
  FileElem *nouveau = calloc(1, sizeof(FileElem));
  nouveau->val = val;


  if(!file->tete)
    {
      file->tete = file->queue = nouveau;
    }
  else
    {
      file->queue->suiv = nouveau;
      file->queue = file->queue->suiv;
    }
}
  \end{lstlisting}
\end{description}

\subsubsection{Défilement}

\begin{description}
  \item[Nom de la fonction:] file\_defiler

  \item[Entrées:] \hfill
    \begin{itemize}
      \item File *file: Une file non vide
    \end{itemize}

 \item[Sortie:] \hfill
   int : La valeur supprimée.

  \item[Description:] \hfill \\ 
    Supprime l'élément en tête de la file.

  \item[Code:] \hfill 
  \begin{lstlisting}

int file_defiler(File *file)
{
  assert(file->tete);
  assert(file->queue);

  int valeur = file->tete->val;


  if(file->tete == file->queue)
    {
      free(file->queue);
      file->tete = file->queue = NULL;
    }
  else
    {
      FileElem *suppr = file->tete;
      file->tete = file->tete->suiv;
      free(suppr);
    }

  return valeur;
}
  \end{lstlisting}


\end{description}



\subsubsection{Copie de la file}
\begin{description}
  \item[Nom de la fonction:] file\_copier

  \item[Entrées:] \hfill
    \begin{itemize}
      \item File *file: Une file
    \end{itemize}

 \item[Sortie:] \hfill
   File* : Une copie de la file

  \item[Description:] \hfill \\ 
    Défile les éléments de la file un par un et les enfile dans deux autres files.
    L'une sera retournée comme copie et on change le pointeur ``file'' pour pointer 
    sur l'autre.

  \item[Code:] \hfill
  \begin{lstlisting}
File file_copier(File *file)
{
  File copie = {NULL, NULL}, file2 = {NULL, NULL};

  while(file->tete)
    {
      int val = file_defiler(file);
      file_enfiler(&file2, val);
      file_enfiler(&copie, val);
    }

  *file = file2;
  return copie;
}
  \end{lstlisting}
\end{description}

\subsubsection{Affichage de la file}
\begin{description}
  \item[Nom de la fonction:] file\_afficher

  \item[Entrées:] \hfill
    \begin{itemize}
      \item File *file: Une file
    \end{itemize}

  \item[Description:] \hfill \\ 
    Défile les éléments de la file un par un et les affiche. Le pointeur ``file''
    et alors changé pour pointer sur la nouvel file.

  \item[Code:] \hfill
  \begin{lstlisting}
void file_afficher(File *file)
{
  File copie = {NULL, NULL};

  printf("[");
  while(file->tete)
    {
      int valeur = file_defiler(file);
      printf("%d", valeur);
      if(file->queue) printf(", ");
      file_enfiler(&copie, valeur);
    }

  printf("]");

  *file = copie;
}
  \end{lstlisting}
\end{description}
\end{document}
